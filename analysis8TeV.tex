\chapter{Searches for Supersymmetry at $\sqrt{s}=8\TeV$}
\label{ch:analysis8TeV}

\section{Event selection}
\label{sec:selection}
Events are selected at the L1 trigger level by requiring at least two
jets with $|\eta|<3$. At the HLT level, events are selected using
dedicated razor algorithms, consisting of a loose selection on \MR and
$\Rtwo$. Razor-specific triggers are used in the HLT in order to avoid
biases on the shapes of distributions from the SM background that are
introduced by requirements on more traditional selection variables
such as $\ETm$.  The razor triggers reject the majority of the SM
background, which mostly appears at low $\Rtwo$ and low $\MR$, while
retaining events in the signal-sensitive regions of the ($\MR$,
$\Rtwo$) plane. Two types of triggers are used: i) a hadronic razor
trigger, which selects events that contain at least two jets with
transverse momentum $\pt>64\GeV$ by applying threshold requirements on
$\Rtwo$, $\MR$, and their product; ii) a muon and electron razor
trigger, which selects events with at least one isolated electron or
muon with $\pt>12\GeV$ in combination with looser requirements on
$\Rtwo$, $\MR$, and their product. The trigger efficiency, evaluated
using a dedicated trigger, is measured to be $(95 \pm 5)\%$ and is
independent of $\Rtwo$ and $\MR$ for the events selected with the
baseline requirements described in Chapter~\ref{ch:kinematic}.

Following the trigger selection, events are required to contain at
least one reconstructed interaction vertex. If more than one vertex is
found, the one with the highest $\pt^2$ sum of associated tracks is
chosen as the interaction point for event reconstruction. Algorithms are
used to remove events with detector- and beam-related noise that can
mimic event topologies with high energy and large $\pt$
imbalance~\cite{Chatrchyan:2011tn,Chatrchyan:2012lia,Khachatryan:2014gga}.

The analysis uses a global event description based on the CMS particle
flow (PF) algorithm~\cite{PF1,PF2}. Individual particles (PF
candidates) are reconstructed by combining the information from the inner
tracker, the calorimeters, and the muon system. Five categories of PF
candidates are defined: muons, electrons, photons (including their
conversions to $\Pep\Pem$ pairs), charged hadrons, and neutral
hadrons. The contamination from other proton-proton collisions in the
same or in neighboring bunch crossings is reduced by discarding the
charged PF candidates not compatible with the interaction point. When
computing lepton isolation and jet energy, the corresponding
contamination from neutral particles is subtracted on average by
applying an event-by-event correction based on the jet-area
method~\cite{jetarea_fastjet,jetarea_fastjet_pu,JME-JINST}.

A ``tight'' lepton identification is used for muons and electrons,
consisting of requirements on isolation and track reconstruction
quality. For electrons, the shape and position of the energy deposit
in the electromagnetic calorimeter is used to further reduce the contamination from
hadrons~\cite{Chatrchyan:2013iaa}. For events with one identified
tight lepton, additional muons or electrons are identified through a
``loose'' lepton selection, characterized by a relaxed isolation
requirement~\cite{Chatrchyan:2013mxa}. Tight leptons are
required to have $\pt>15$\GeV and loose leptons $\pt>10$\GeV.

Jets are reconstructed by clustering the PF candidates with the
\textsc{FastJet}~\cite{fastjet} implementation of the anti-\kt~\cite{antikt} algorithm with the distance parameter $R=0.5$. We
select events containing at least two jets with $\pt>80$\GeV and
$\abs{\eta}<2.4$, representing a tighter version of the L1 jet selection criterion. The $\pt$
imbalance in the event, $\ptvecmiss$, is the
negative of the sum of the $\ptvec$ of the PF candidates in the
event. Its magnitude is referred to as $\ETm$. For each event, the $\ptvecmiss$ and the
four-momenta of all the jets with $\pt>40$\GeV and $\abs{\eta}<2.4$ are
used to compute the razor variables, as described in
Section~\ref{sec:razVar}.

The medium working point of the combined secondary vertex
algorithm~\cite{btag8TeV} is used for b-jet tagging. The \PQb-tagging
efficiency and mistag probability are measured from data control
samples as a function of the jet $\pt$ and $\eta$. Correction factors
are derived for Monte Carlo (MC) simulations through comparison of the
measured and simulated \PQb-tagging efficiencies and mistag rates found
in these control samples~\cite{btag8TeV}.

Events with no \PQb-tagged jet are discarded, a criterion motivated by
the natural SUSY signatures described in Section~\ref{sec:sms}. A tighter
requirement ($\geq$2 \PQb-tagged jets) is imposed on events without an
identified tight lepton and fewer than four jets. This requirement reduces the
expected background from SM production of $\cPZ(\to\nu\bar\nu)$+jets
events to a negligible level.


\section{Box definitions}
\label{sec:razVar}

The selected events are categorized into the different razor boxes according to
their event content as shown in Table~\ref{tab:boxDef}. In the table,
the boxes are listed according to the filling order, from the first
(at the top of the table) to the last (at the bottom). If an event
satisfies the requirements of two or more boxes, the event is assigned
to the first listed box to ensure the boxes correspond to disjoint samples.

The events in the single-lepton and two-lepton boxes are recorded
using the electron and muon razor trigger. The remaining two boxes, generically
referred to as ``hadronic'' boxes, contain events recorded using the
hadronic razor trigger.

In the two-lepton boxes, the ($\MR$, $\Rtwo$)
distribution of events with at least one \PQb-tagged jet is studied. For
the other boxes, the data are binned according to the \PQb-tagged jet
multiplicity: 1 \PQb-tag, 2 \PQb-tags, and $\geq$3 \PQb-tags.

\begin{table*}[ht!]
\centering
 \caption{Kinematic and multiplicity requirements defining the nine
 razor boxes. Boxes are listed in order of event filling priority.
 \label{tab:boxDef}}
\resizebox{\textwidth}{!}{
\begin{tabular}{ccccc}
Box & Lepton & \PQb-tag & Kinematic & Jet \\
\hline
 \multicolumn{5}{c}{Two-lepton boxes}\\
\hline
\multirow{2}{*}{MuEle} & $\geq$1 tight electron and & \multirow{6}{*}{$\geq$1 \PQb-tag} & \multirow{2}{*}{} & \multirow{6}{*}{$\geq$2 jets}\\
& $\geq$1 loose muon & & & \\
\cline{1-2}
\multirow{2}{*}{MuMu} & $\geq$1 tight muon and & & ($\MR >300$\GeV and $\Rtwo > 0.15$) and & \\
& $\geq$1 loose muon & & ($\MR > 350$\GeV  or $\Rtwo > 0.2$) & \\
\cline{1-2}
\multirow{2}{*}{EleEle} & $\geq$1 tight electron and & & & \\
& $\geq$1 loose electron& & & \\
\hline
\multicolumn{5}{c}{Single-lepton boxes}\\
\hline
MuMultiJet & 1 tight muon & \multirow{4}{*}{$\geq$1 \PQb-tag} & & \multirow{2}{*}{$\geq$4 jets} \\
%\cline{1-2}
EleMultiJet &1 tight electron & & ($\MR > 300$\GeV and $\Rtwo > 0.15$) and & \\
%\cline{1-2}
\cline{5-5}
MuJet & 1 tight muon & & ($\MR > 350$\GeV or $\Rtwo > 0.2$) & \multirow{2}{*}{2 or 3 jets}\\
%\cline{1-2}
EleJet & 1 tight electron & & &  \\
\hline
\multicolumn{5}{c}{Hadronic boxes}\\
\hline
MultiJet & none & $\geq$1 \PQb-tag & ($\MR > 400$\GeV and $\Rtwo > 0.25$) and &$\geq$4 jets\\
%\cline{1-3}
%\cline{5-5}
$\geq$2 \PQb-tagged jet & none & $\geq$2 \PQb-tag &  ($\MR > 450$\GeV or $\Rtwo > 0.3$) & 2 or 3 jets\\
\end{tabular}}
\end{table*}

A baseline kinematic requirement is applied to define the region in
which we search for a signal:
\begin{itemize}
\item $\MR>400$\GeV and $\Rtwo>0.25$ for the hadronic boxes;
\item $\MR>300$\GeV and $\Rtwo>0.15$ for the other boxes.
\end{itemize}
The tighter baseline selection for the hadronic boxes is a consequence
of the tighter threshold used for the hadronic razor trigger. The
kinematic plane defined by the baseline selection is divided into three
regions (see Fig.~\ref{fig:regions}):
\begin{itemize}
\item Low \MR sideband: $400<\MR<550$\GeV
 and $\Rtwo>0.30$ for the hadronic boxes;
 $300<\MR<450$\GeV and $\Rtwo>0.20$ for the other
 boxes.
\item Low  \Rtwo sideband: $\MR>450$\GeV and
  $0.25<\Rtwo<0.30$ for the hadronic boxes;
  $\MR>350$\GeV and $0.15<\Rtwo<0.20$ for the other
  boxes.
\item Signal-sensitive region: $\MR>550$\GeV and
 $\Rtwo>0.30$ for the hadronic boxes; $\MR>450$\GeV
 and $\Rtwo>0.20$ for the other boxes.
\end{itemize}
The bottom left corner of the razor plane, not included in any of the
three regions, is excluded from the analysis. Given this selection,
the multijet background from quantum chromodynamics processes is
reduced to a negligible level due to the fact that these processes
typically peak at $\Rtwo\approx0$ and fall exponentially for
larger values of $\Rtwo$~\cite{razorPRL,razorPRD}.

\begin{figure}[ht!]
\centering
\includegraphics[width=0.49\textwidth]{figs/analysis8TeV/SidebandL_MultiJet.pdf}
\includegraphics[width=0.49\textwidth]{figs/analysis8TeV/SidebandL_Mu.pdf}
\caption{\label{fig:regions} Definition of the sideband and the
 signal-sensitive regions used in the analysis, for (left) the hadronic
 boxes and (right) the other boxes.}
\end{figure}

\section{Modeling of the standard model backgrounds}
\label{sec:bmodel}
Under the hypothesis of no contribution from new-physics processes,
the event distribution in the considered portion of the
($\MR$, $\Rtwo$) plane can be described by the sum of
the contributions from SM $\cPV$+jets events (where
$\cPV$ indicates a $\PW$ or $\cPZ$ boson) and SM top quark-antiquark and
single-top events, where the events with a top quark are generically
referred to as the $\ttbar$ contribution. Based on MC studies, the
contributions from other processes are determined to be
negligible.

We study each of these processes using MC samples, generated with the
\MADGRAPH v5
simulation~\cite{Alwall:2011uj,Alwall:2014hca}. Parton shower and
hadronization effects are included by matching events to the \PYTHIA v6.4.26 simulation~\cite{Sjostrand:2006za} using the MLM
algorithm~\cite{Hoche:2006ph}. The events are processed by a
\GEANT-based~\cite{G4} description of the CMS apparatus in order to
account for the response of the detector.

Once normalized to the NLO inclusive cross
section and the integrated luminosity, the absolute yield of the
$\cPV$+jets events contribution satisfying the event selection is found
to be negligible in all of the two-lepton boxes. In the remaining boxes,
its contribution to the total SM background is found to be
approximately 25\%. The contribution of $\cPV$+jets events in
the $\geq$2 \PQb-tag and the $\geq$4 jet sample is found to be
negligible. The remainder of the background in each box originates
from $\ttbar$ events.

\subsection{Empirical Razor Function}
Based on the study of the data collected at $\sqrt{s}=7\TeV$ and the
corresponding MC samples~\cite{razorPRL,razorPRD}, the two-dimensional
probability density function
$P_\mathrm{SM}(\MR,\Rtwo)$ for each SM process is
found to be well described by the empirical function
\begin{equation}
 f(\MR,\Rtwo) =  \bigl[b(\MR-{\MRz})^{1/n}(\Rtwo-{\Rtwoz})
  ^{1/n}-1\bigr]\re^{-bn(\MR-{\MRz})^{1/n}(\Rtwo-{\Rtwoz})
    ^{1/n}} ,
\label{eq:razFun}
\end{equation}
where $b$, $n$, $\MRz$, and $\Rtwoz$ are free
parameters of the background model. 

For $n=1$, this function recovers the two-dimensional exponential function used for previous
studies~\cite{razorPRL,razorPRD}. The original motivation is detailed
in the cited papers, but a quick summary follows. There is an observed
correlation between the two razor variables such that after a baseline
selection $\MR>\MR^{\mathrm{min}}$ and $\Rtwo>\Rtwo_{\mathrm{min}}$, the distributions of the SM backgrounds
exhibit an exponential behavior in $\Rtwo$ ($\MR$) when integrated over
$(\MR)$, ($\Rtwo$):
\begin{align}
 \int^{\infty}_{\Rtwo_{\mathrm{min}}} P_\mathrm{SM}(\MR,\Rtwo)
  \mathrm{d}\Rtwo &\propto e^{-(r_0 + r_1\Rtwo_{\mathrm{min}})M_R}~,\\
 \int^{\infty}_{\MR^{\mathrm{min}}} P_\mathrm{SM}(\MR,\Rtwo)
  \mathrm{d}\MR &\propto  e^{-(m_0+ m_1\MR^{\mathrm{min}})\Rtwo}~,
\label{eq:2dcorrelation}
\end{align}
where $r_0$, $r_1$, $m_0$, and $m_1$ are interrelated exponential parameters.
This behavior for QCD multijets background is illustrated in
Fig.~\ref{fig:qcdfit}. The empricial function in Eqn.~\ref{eq:razFun} with
$n=1$ perfectly replicates this behavior and the exponential parameters can be identified with
the empirical function's parameters, namely $r_0 = - b\Rtwoz$, $m_0 = - b\MRz$, and $r_1=m_1=b$.

\begin{figure}[tb!]
\centering
\includegraphics[width=0.49\textwidth]{figs/analysis8TeV/qcd-mr-prd.pdf}
\includegraphics[width=0.49\textwidth]{figs/analysis8TeV/qcd-rsq-prd.pdf}\\
\includegraphics[width=0.49\textwidth]{figs/analysis8TeV/qcd-slopeMR-prd.pdf}
\includegraphics[width=0.49\textwidth]{figs/analysis8TeV/qcd-slopeR-prd.pdf}
\caption{QCD multijet events collected by CMS at $\sqrt{s}=7\TeV$
  demonstrate the two-dimensional correlation between $\MR$ and
  $\Rtwo$ that motivates the original functional form.\label{fig:qcdfit}}
\end{figure}

To account for the possibility of non-exponential tails of the SM
backgrounds, the $\sqrt{s}=7\TeV$ search invoked two copies of the
empirical function with $n=1$ to model each SM background. 
For the $\sqrt{s}=8\TeV$ search, we take a different approach by using
only one instance of the function, but allowing the $n$ parameter to deviate
from $1$. Fig.~\ref{fig:twoexp} illustrates the similarity between using two exponential components and using one
instance of the generalized function.

\begin{figure}[tb!]
\centering
\includegraphics[width=0.8\textwidth,clip=true,viewport= 0 70 600 410]{figs/analysis8TeV/twoexp.pdf}
\caption{Two exponential components with $n=1$ and their sum are shown in blue compared with
  a single modified exponential with $n=3$ in black.\label{fig:twoexp}}
\end{figure}


One of the benefits of this functional form is that it is analytically
integrable. By providing the analytical integral to \textsc{RooFit},
we avoid using \textsc{RooFit}'s multi-dimensional numerical
integration, which is costly in terms of function evaluations and may be inaccurate~\cite{Anderson:2007}~\cite{Press:1992:NRC:148286}.
In particular, the one-dimensional and two-dimensional integrals of
the function are
\begin{align}
 \int^{\Rtwo_{\mathrm{max}}}_{\Rtwo_{\mathrm{min}}} f(\MR,\Rtwo)
  \mathrm{d}\Rtwo &=  
\exp\left(-bn(\MR-\MRz)^{1/n}(\Rtwo_{\mathrm{max}}-\Rtwoz)^{1/n}\right))~~~~~~~~~~~~~~~~~~~~\nonumber\\
&\times\exp\left(-bn(\MR-\MRz)^{1/n}(\Rtwo_{\mathrm{min}}-\Rtwoz)^{1/n}\right) \nonumber\\
&\times\bigg(\exp\left(-b n (\MR-\MRz)^{1/n}
   (\Rtwo_{\mathrm{max}}-\Rtwoz)^{1/n}\right) (\Rtwo_{\mathrm{min}}-\Rtwoz)\nonumber\\
& -\exp\left(-b n (\MR-\MRz)^{1/n} (\Rtwo_{\mathrm{min}}-\Rtwoz)^{1/n}\right)
   (\Rtwo_{\mathrm{max}}-\Rtwoz)\bigg)~,
\end{align}
and
\begin{align}
 \int^{\MR^{\mathrm{max}}}_{\MR^{\mathrm{min}}}
  \int^{\Rtwo_{\mathrm{max}}}_{\Rtwo_{\mathrm{min}}} f(\MR,\Rtwo)
  \mathrm{d}\Rtwo \mathrm{d}\MR &=  n (b n)^{-n} \bigg(\Gamma \left(n,b n (\MRz-\MR^{\mathrm{max}})^{1/n}
   (\Rtwoz-\Rtwo_{\mathrm{max}})^{1/n}\right)\nonumber\\
&- \Gamma \left(n,b n
   (\MR^{\mathrm{min}}-\MRz)^{1/n}
   (\Rtwo_{\mathrm{max}}-\Rtwoz)^{1/n}\right)\nonumber\\
&-\Gamma \left(n,b n
   (\MR^{\mathrm{max}}-\MRz)^{1/n}
   (\Rtwo_{\mathrm{min}}-\Rtwoz)^{1/n}\right)\nonumber\\
&+\Gamma \left(n,b n
   (\MR^{\mathrm{min}}-\MRz)^{1/n}
   (\Rtwo_{\mathrm{min}}-\Rtwoz)^{1/n}\right) \bigg)~,
\label{eq:razFunIntegrals}
\end{align}
respectively, where $\Gamma(a,x)$ is the incpmplete gamma function:
\begin{equation}
 \Gamma(a,x)=\int_{x}^{\infty}t^{a-1}e^{-t}\mathrm{d}t~.
\end{equation}

The shape of the empirical function is determined through a \textsc{RooFit}-based extended and unbinned
maximum likelihood fit to the data~\cite{Verkerke:2003ir}. Two kinds
of fit are performed: (i)~a sideband-only fit, which is extrapolated
to the signal region in order to test for the presence of a signal
(discussed in the remainder of this section), and (ii)~a simultaneous
fit to the signal and sideband regions, performed both under the
background-only and background-plus-signal hypotheses, which is used
for the interpretation of the results (Section~\ref{sec:limit}). In both cases, the empirical function is
found to adequately describe the SM background in each of the boxes,
for each \PQb-tagged jet multiplicity value.

The SM background-only likelihood function for the two-lepton boxes is written as:
\begin{equation}
\mathcal{L}(\text{data}|\Theta) = \frac{\re^{-N_\mathrm{SM}}}{N!} \prod_{i=1}^{N} N_\mathrm{SM}
 P_\mathrm{SM}({\MR}_{(i)},{\Rtwo}_{(i)}),
\label{eq:Lik1btag}
\end{equation}
where $P_\mathrm{SM}(\MR,\Rtwo)$ is the empirical function in
Eq.~(\ref{eq:razFun}) normalized to unity, $N_{SM}$ is the
corresponding normalization factor, $\Theta$ is the set of
background shape and normalization parameters, and the product runs
over the $N$ events in the data set. The same form of the
likelihood is used for the other boxes, for each \PQb-tagged jet
multiplicity. The total likelihood in these boxes is computed as the
product of the likelihood functions for each \PQb-tagged jet
multiplicity.

The fits are performed independently for each box and simultaneously
across the \PQb-tagged jet multiplicity bins. Common background shape
parameters ($b$, ${\MR}^0$, $\Rtwoz$, and $n$) are used
for the 2 \PQb-tag and $\geq$3 \PQb-tag bins, since no substantial
difference between the two distributions is observed on large samples
of $\ttbar$ and $\cPV$+jets MC events. A difference is observed
between 1 \PQb-tag and $\geq$2 \PQb-tag samples, due to the observed
dependence of the \PQb-tagging efficiency on the jet $\pt$. Consequently,
the shape parameters for the 1 \PQb-tag bins are allowed to differ
from the corresponding parameters for the $\geq$2 \PQb-tag bins. The
background normalization parameters for each \PQb-tagged jet multiplicity
bin are also treated as independent parameters.

The background shape parameters are estimated from the events in the
two sidebands (Section~\ref{sec:razVar}). This shape is then used to
derive a background prediction in the signal-sensitive region:
$30\,000$ alternative sets of background shape parameters are generated
from the covariance matrix returned by the fit. An ensemble of
pseudo-experiment data sets is created, generating random
($\MR$, $\Rtwo$) pairs distributed according to each
of these alternative shapes. For each bin of the signal-sensitive
region, the distribution of the predicted yields in each
pseudo-experiment is compared to the observed yield in data in order
to quantify the agreement between the background model and the
observation. The agreement, described as a two-sided p-value, is then
translated into the corresponding number of standard deviations for a
normal distribution. The p-value is computed using the probability
density as the ordering principle. The observed numbers of standard
deviations in the two-lepton boxes are shown in
Fig.~\ref{fig:FrenchFlagDilep}, as a function of \MR and
$\Rtwo$. Positive and negative significance correspond to
regions where the observed yield is respectively larger and smaller
than the predicted one. Light gray areas correspond to empty bins with
less than one event expected on average. Similar results for the
one-lepton and hadronic boxes are shown in
Figs.~\ref{fig:FrenchFlagLep} and
\ref{fig:FrenchFlagHad}. Figures~\ref{fig:Proj1DDilep}--\ref{fig:Proj1DHad}
illustrate the extrapolation of the fit results to the full
($\MR$, $\Rtwo$) plane, projected onto  \Rtwo and \MR and summed over the \PQb-tagged jet multiplicity
bins. No significant deviation of data from the SM background
predictions is observed.

\begin{figure}[tb!]
\centering
\includegraphics[width=0.49\textwidth]{figs/analysis8TeV/nSigmaLog_MuEle.pdf}
\includegraphics[width=0.49\textwidth]{figs/analysis8TeV/nSigmaLog_MuMu.pdf}
\includegraphics[width=0.49\textwidth]{figs/analysis8TeV/nSigmaLog_EleEle.pdf}
\caption{Comparison of the expected background and the observed yield
  in the (upper left) MuEle, (upper right) MuMu, and (bottom)
  EleEle boxes. A probability density function is derived for the
  bin-by-bin yield using pseudo-experiments, sampled from the output
  of the corresponding sideband fit. A two sided p-value is computed
  comparing the observed yield to the distribution of background yield
  from pseudo-experiments. The p-value is translated into the
  corresponding number of standard deviations, quoted in each bin and
  represented by the bin-filling color. Positive and negative
  significance correspond to regions where the observed yield is
  respectively larger and smaller than the predicted one. The white areas
  correspond to bins in which a difference smaller than 0.1 standard
  deviations is observed. The gray areas correspond to empty bins with
  less than one background event expected on average. The dashed lines
  represent the boundaries between the sideband and the signal
  regions.\label{fig:FrenchFlagDilep}}

\end{figure}

\begin{figure*}[tb!]
\centering
\includegraphics[width=0.49\textwidth]{figs/analysis8TeV/nSigmaLog_EleJet.pdf}
\includegraphics[width=0.49\textwidth]{figs/analysis8TeV/nSigmaLog_EleMultiJet.pdf}
\includegraphics[width=0.49\textwidth]{figs/analysis8TeV/nSigmaLog_MuJet.pdf}
\includegraphics[width=0.49\textwidth]{figs/analysis8TeV/nSigmaLog_MuMultiJet.pdf}
\caption{Comparison of the expected background and the observed yield
  in (upper left) the EleJet, (upper right) the EleMultiJet, (lower left) the MuJet, and (lower right) the MuMultiJet
  boxes. A detailed explanation is given in the caption of
  Fig.~\ref{fig:FrenchFlagDilep}.\label{fig:FrenchFlagLep}}
\end{figure*}

\begin{figure}[tb!]
\centering
\includegraphics[width=0.49\textwidth]{figs/analysis8TeV/nSigmaLog_Jet2b.pdf}
\includegraphics[width=0.49\textwidth]{figs/analysis8TeV/nSigmaLog_MultiJetFITS.pdf}
\caption{Comparison of the expected background and the observed yield
  in the $\geq$2 \PQb-tagged jet box (left) and the MultiJet box
  (right). A detailed explanation is given in the caption of
  Fig.~\ref{fig:FrenchFlagDilep}.\label{fig:FrenchFlagHad}}

\end{figure}

\begin{figure*}[tb!]
\centering
\includegraphics[width=0.49\textwidth]{figs/analysis8TeV/MR_ElectronHad-Run2012ABCD_Sideband_MuEle.pdf}
\includegraphics[width=0.49\textwidth]{figs/analysis8TeV/RSQ_ElectronHad-Run2012ABCD_Sideband_MuEle.pdf}
\includegraphics[width=0.49\textwidth]{figs/analysis8TeV/MR_MuHad-Run2012ABCD_Sideband_MuMu.pdf}
\includegraphics[width=0.49\textwidth]{figs/analysis8TeV/RSQ_MuHad-Run2012ABCD_Sideband_MuMu.pdf}
\includegraphics[width=0.49\textwidth]{figs/analysis8TeV/MR_ElectronHad-Run2012ABCD_Sideband_EleEle.pdf}
\includegraphics[width=0.49\textwidth]{figs/analysis8TeV/RSQ_ElectronHad-Run2012ABCD_Sideband_EleEle.pdf}
\caption{Projection of the sideband fit result in the (upper row) MuEle, (middle row)
  MuMu, and (lower row) EleEle boxes on \MR (left) and
   \Rtwo (right), respectively. The fit is performed
  in the sideband regions and extrapolated to the signal-sensitive
  region. The solid line and the filled band represent the total
  background prediction and its uncertainty. The points and the band
  in the bottom panel represent the data-to-prediction ratio and the
  prediction uncertainty, respectively.\label{fig:Proj1DDilep}}
\end{figure*}

\begin{figure*}[tb!]
\centering
\includegraphics[width=0.49\textwidth]{figs/analysis8TeV/MR_MuHad-Run2012ABCD_Sideband_MuJet.pdf}
\includegraphics[width=0.49\textwidth]{figs/analysis8TeV/RSQ_MuHad-Run2012ABCD_Sideband_MuJet.pdf}
\includegraphics[width=0.49\textwidth]{figs/analysis8TeV/MR_MuHad-Run2012ABCD_Sideband_MuMultiJet.pdf}
\includegraphics[width=0.49\textwidth]{figs/analysis8TeV/RSQ_MuHad-Run2012ABCD_Sideband_MuMultiJet.pdf}
\caption{Projection of the sideband fit result in the MuJet box on (upper left)
  \MR and (upper right) $\Rtwo$, and of the sideband fit
  result in the MuMultiJet box on (lower left) \MR and (lower right)
  $\Rtwo$. The fit is performed in the sideband regions and
  extrapolated to the signal-sensitive region. The solid line and the
  filled band represent the total background prediction and its
  uncertainty. The dashed and dot-dashed lines represent the
  background shape for 1 \PQb-tag and $\geq$2 \PQb-tag events,
  respectively. The points and the band in the bottom panel represent
  the data-to-prediction ratio and the prediction uncertainty,
  respectively.\label{fig:Proj1DMu}}

\end{figure*}

\begin{figure*}[tb!]
\centering
\includegraphics[width=0.49\textwidth]{figs/analysis8TeV/MR_ElectronHad-Run2012ABCD_Sideband_EleJet.pdf}
\includegraphics[width=0.49\textwidth]{figs/analysis8TeV/RSQ_ElectronHad-Run2012ABCD_Sideband_EleJet.pdf}
\includegraphics[width=0.49\textwidth]{figs/analysis8TeV/MR_ElectronHad-Run2012ABCD_Sideband_EleMultiJet.pdf}
\includegraphics[width=0.49\textwidth]{figs/analysis8TeV/RSQ_ElectronHad-Run2012ABCD_Sideband_EleMultiJet.pdf}
\caption{Projection of the sideband fit result in the EleJet box on
  (upper left) \MR and (upper right) $\Rtwo$, and projection of the
  sideband fit result in the EleMultiJet box on (lower left) \MR and
  (lower right) $\Rtwo$. A detailed explanation is given in the caption
  of Fig.~\ref{fig:Proj1DMu}.\label{fig:Proj1DEle}}
\end{figure*}

\begin{figure*}[tb!]
\centering
\includegraphics[width=0.49\textwidth]{figs/analysis8TeV/MR_HT-HTMHT-Run2012ABCD_Sideband_Jet2b.pdf}
\includegraphics[width=0.49\textwidth]{figs/analysis8TeV/RSQ_HT-HTMHT-Run2012ABCD_Sideband_Jet2b.pdf}
\includegraphics[width=0.49\textwidth]{figs/analysis8TeV/MR_HT-HTMHT-Run2012ABCD_Sideband_MultiJet.pdf}
\includegraphics[width=0.49\textwidth]{figs/analysis8TeV/RSQ_HT-HTMHT-Run2012ABCD_Sideband_MultiJet.pdf}
\caption{Projection of the sideband fit result in the $\geq$2 \PQb-tagged jet
  box on (upper left) \MR and (upper right) $\Rtwo$, and projection of
  the sideband fit result in the MultiJet box on (lower left) \MR   and (lower right) $\Rtwo$. A detailed explanation is given in the
  caption of Fig.~\ref{fig:Proj1DMu}.\label{fig:Proj1DHad}}
\end{figure*}

To demonstrate the discovery potential of this analysis, we apply the
background-prediction procedure to a simulated signal-plus-background
MC sample. Figure~\ref{fig:T1bbbbsignalinj} shows the \MR and  \Rtwo distributions of SM background events and T1bbbb
events (Section~\ref{sec:sms}). The gluino and LSP masses are set
respectively to 1325\GeV and 50\GeV, representing a new-physics
scenario near the expected sensitivity of the analysis. A
signal-plus-background sample is obtained by adding the two
distributions of Fig.~\ref{fig:T1bbbbsignalinj}, assuming an
integrated luminosity of 19.3\fbinv and a gluino-gluino production
cross section of 0.02\unit{pb}, corresponding to 78 expected signal events
in the signal-sensitive region. The agreement between the background
prediction from the sideband fit and the yield of the
signal-plus-background pseudo-experiments is displayed in
Fig.~\ref{fig:FFsigma0p02}. The contribution of signal events to the
sideband region has a negligible impact on the determination of the
background shape, while a disagreement is observed in the
signal-sensitive region, characterized as an excess of events
clustered around $\MR\approx1300$\GeV. The excess indicates
the presence of a signal, and the position of the excess in the
$\MR$ variable provides information about the underlying SUSY
mass spectrum.

\begin{figure}[htb!]
\centering
\includegraphics[width=0.49\textwidth]{figs/analysis8TeV/SMbkgd_FF.pdf}
\includegraphics[width=0.49\textwidth]{figs/analysis8TeV/T1bbbb_1325_50_FF.pdf}
\caption{Distribution of (left) simulated SM background events and (right)
  T1bbbb gluino-gluino events in the MultiJet box. Each $\sGlu$ is
  forced to decay to a \bbbar pair and a $\chiz_1$,
  assumed to be the stable LSP. The $\sGlu$ and $\chiz_1$ masses are
  fixed to 1325\GeV and 50\GeV,
  respectively.\label{fig:T1bbbbsignalinj}}

\end{figure}

\begin{figure}[htb!]
\centering
\includegraphics[width=0.49\textwidth]{figs/analysis8TeV/MR_T1bbbb_0p02_MultiJet.pdf}
\includegraphics[width=0.49\textwidth]{figs/analysis8TeV/RSQ_T1bbbb_0p02_MultiJet.pdf}
\includegraphics[width=0.49\textwidth]{figs/analysis8TeV/nSigmaLog_MultiJet.pdf}
\caption{Result of the fit to the sideband events of a
  signal-plus-background MC sample, corresponding to the gluino model
  whose distribution is shown in Fig.~\ref{fig:T1bbbbsignalinj}. A
  gluino-gluino production cross section of 0.02\unit{pb} is assumed. The
  one-dimensional projections on (upper left) \MR and (upper right)
   \Rtwo are shown, together with (bottom) the agreement between
  the observed yield and the prediction from the sideband fit as a
  function of  \Rtwo and $\MR$. This agreement is
  evaluated from a two-sided p-value using an ensemble of
  background-only pseudo-experiments as described in
  Section~\ref{sec:bmodel}.\label{fig:FFsigma0p02}}
\end{figure}
