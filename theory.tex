\section{The Standard Model of Particle Physics}

The standard model (SM) of particle physics is a description of nature
composed of two main ingredients:
\begin{enumerate}[(a)]
\item a list of known particles and their quantum properties, and
\item an inventory of the interactions of these particles through the fundamental forces
(except for gravity): the strong nuclear force, the weak nuclear force, and the electromagnetic
force. 
\end{enumerate}
It is a renormalizable quantum field theory based on a gauge
symmetry~\cite{tHooft:419894}. A fundamental consequence of relativity and quantum mechanics
is that there are only two types of particles: those with integer
spins, whose wave functions are symmetric under particle exchange, called \emph{bosons}, and those
with half-integer spin, whose wave functions are anti-symmetric under
particle exchange, called \emph{fermions}~\cite{PhysRev.58.716}. The forces in
the SM arise due to the exchange of spin-$1$ bosons among the
spin-$\frac{1}{2}$ fermions that make up matter. 

Each factor in the gauge symmetry group $\mathrm{SU(3)}_{\mathrm{C}}\times
\mathrm{SU(2)}_{\mathrm{L}}\times\mathrm{U(1)}_Y$ corresponds to a fundamental force, represented by a gauge field, whose excitations are
the gauge bosons that act as force carriers:
\begin{center}
\begin{tabular}{ccccc}
$\mathrm{SU(3)}_{\mathrm{C}}$ &$\times$& $\mathrm{SU(2)}_{\mathrm{L}}$
  &$\times$& $\mathrm{U(1)}_Y$\\
 $\downarrow$&&$\downarrow$&&$\downarrow$\\
 $G_{\mu}^{\alpha}$&&$W^a_{\mu}$&&$B_{\mu}$\\
 $\alpha=1,...,8$&&$a=1,2,3$&&
\end{tabular}
\end{center}
There are eight bosons, called gluons, represented by the fields
$G_{\mu}^{\alpha}$ and associated with the factor
$\mathrm{SU(3)}_{\mathrm{C}}$. The three bosons represented by
the fields  $W^{a}_{\mu}$ and associated with the factor
$\mathrm{SU(2)}_{\mathrm{L}}$, and the boson represented by the field
$B_{\mu}$ and associated with the factor $\mathrm{U(1)}_Y$ mix to form the
$\PW^{\pm}$ boson, $\PZ$ boson, and the photon $\Pgg$.

The matter fields are fermions, which fall into two
categories: the quarks, $u$, $d$, $c$, $s$, $t$, and $b$, which participate in the
strong interactions, and the leptons, $e$, $\mu$, $\tau$, $\nu_e$,
$\nu_{\mu}$, and $\nu_{\tau}$, which do
not. The fermions also naturally fit into three generations of matter,
as displayed in Fig.~\ref{fig:standardmodel}. 

The second ingredient of the SM, an inventory of the interactions between the
particles, is given by the Langrangian density, 
\begin{align}
\mathcal{L}_{\mathrm{SM}} &= -\frac{1}{4}B_{\mu\nu}B^{\mu\nu} -\frac{1}{4}W^{a}_{\mu\nu}W^{a\mu\nu} - \frac{1}{4}G^{\alpha}_{\mu\nu}G^{\alpha\mu\nu}
  & \mathrm{(gauge~terms)}\nonumber\\
& +\bar\ell_L\tilde\sigma^{\mu}iD_{\mu}\ell_L +
   \bar e_R\sigma^{\mu}iD_{\mu}e_R + \bar v_R
   \sigma^{\mu}iD_{\mu}\nu_R + (\mathrm{h.c.})& \mathrm{(lepton~kinetic~terms)}\nonumber\\
& +\bar q_L\tilde\sigma^{\mu}iD_{\mu}q_L +
   \bar u_R\sigma^{\mu}iD_{\mu}u_R + \bar d_R
   \sigma^{\mu}iD_{\mu}d_R + (\mathrm{h.c.})& \mathrm{(quark~kinetic~terms)}\nonumber\\
& +\mathcal L_{\mathrm{Higgs}} +\mathcal L_{\mathrm{Yukawa}} &  \mathrm{(Higgs~and~Yukawa~terms)}
\label{eqn:lsm}
\end{align}
where $D_{\mu}$ is the gauge-covariant derivative, $q_L =
\binom{u_L}{d_L}$ and $\ell_L = \binom{e_L}{\nu_L}$ are
$\mathrm{SU(2)}_{\mathrm{L}}$ doublets, and the three-component
generation indices are suppressed.
%$e_i=(e,\mu,\tau), \nu_i=(\nu_e,\nu_{\mu},\nu_{\tau}), u_i=(u,c,t),
%d_i=(d,s,b)$. 
Tab.~\ref{tab:representations} summarizes another way to visualize the
interactions of the particles, which are the representations in which the matter fields transform under the SM
gauge group. The Higgs terms $\mathcal L_{\mathrm{Higgs}}$ are discussed in
Sec.~\ref{sec:ewsb}, while the Yukawa terms $\mathcal
L_{\mathrm{Yukawa}}$ are discussed in Sec.~\ref{sec:fermionmasses}.

\begin{figure}
\centering
\includegraphics[width=0.7\textwidth]{figs/theory/standardmodel.pdf}
\caption{\label{fig:standardmodel} The particles in the standard model.}
\end{figure}

\begin{table}
\centering
\begin{tabular}{c|ccc}
&$\mathrm{SU(3)}_{\mathrm{C}}$&$\mathrm{SU(2)}_{\mathrm{L}}$&$\mathrm{U(1)}_Y$ \\\hline
$q$ & $\mathbf{3}$ & $\mathbf{2}$ & $1/6$\\
$\bar u_R$ & $\mathbf{\bar 3}$ & $\mathbf{1}$ & $-2/3$\\
$\bar d_R$ & $\mathbf{\bar 3}$ & $\mathbf{1}$ & $1/3$\\
$\ell$ & $\mathbf{1}$ & $\mathbf{2}$ & $-1/2$\\
$\bar e_R$ & $\mathbf{1}$ & $\mathbf{1}$ & $1$\\\hline
$h$ & $\mathbf{1}$ & $\mathbf{2}$ & $1/2$
\end{tabular}
\caption{\label{tab:representations} Table summarizing the
    representations in which the matter fields transform under the standard
    model gauge group. $\mathbf{n}$ ($\mathbf{\bar n}$) is the
    fundamental (antifundamental) representation
    of $\mathrm{SU(n)}$. For the $\mathrm{U(1)}_Y$ factor, the
    representations are labeled by the weak hypercharge $Y$. The electric charge is given by $Q = T_{3L}+Y$. }
\end{table}


\section{Electroweak Symmetry Breaking}
\label{sec:ewsb}
A central feature of gauge theories is that the gauge bosons are
massless due to the fact that the gauge symmetry forbids explicit mass
terms in the Lagrangian. In 1964, it was proposed that
\emph{spontaneous symmetry breaking} could be achieved in gauge
theories through the introduction of a scalar
field~\cite{PhysRevLett.13.321,HIGGS1964132,PhysRevLett.13.508,PhysRevLett.13.585,PhysRev.145.1156,PhysRev.155.1554}. Spontaneous symmetry breaking
means the equations of the dynamics are exactly symmetric, but they
admit solutions that are not. In the SM, the mechanism of electroweak symmetry breaking
is a framework to keep the structure of gauge symmetry and
interactions at high energy, and still generate the observed masses
of the $\PW^{\pm}$ and $\PZ$ gauge
bosons~\cite{PhysRevLett.19.1264,GLASHOW1961579,Salam:1968rm}. 

The part of the Lagrangian that accomplishes this is: 
\begin{align}
\mathcal L_{\mathrm{Higgs}} &= (D_{\mu}\Phi)^{\dagger}(D^{\mu}\Phi) -
V(\Phi)~;& V(\Phi) &= -\mu^2\Phi^{\dagger}\Phi +
\lambda(\Phi^{\dagger}\Phi)^2~,
\label{eqn:Lhiggs}
\end{align}
where the field $\Phi$ is a complex, spin-$0$, self-interacting
$\mathrm{SU(2)}_{\mathrm{L}}$ doublet with weak hypercharge $Y=1/2$:
\begin{equation}
\Phi = \left(\begin{matrix} \phi^{+}\\\phi^0\end{matrix} \right)~.
\end{equation}
If $\mu^2>0$, then the potential will have a ``Mexican hat'' shape, illustrated in
\ref{fig:mexicanhat}, and the minimum of the potential will occur at a value of the field that is not $\mathrm{SU(2)}_{\mathrm{L}}\times\mathrm{U(1)}_Y$
invariant. Due to this, $\Phi$ acquires a nonzero vacuum
expectation value (\emph{vev}), corresponding to the minimum of the potential,
\begin{align}
\langle\Phi\rangle_0&\equiv \langle 0|\Phi|0\rangle =
\frac{1}{\sqrt{2}}U(x)\left(\begin{matrix} 0\\v\end{matrix} \right)~;&v &= \sqrt{\frac{\mu^2}{\lambda}};
\end{align}
where $U(x)$ is a unitary transformation that rotates the field
to the other degenerate solutions. Since the vev is still symmetric under a $\mathrm{U(1)}$ subgroup of the full
electroweak symmetry, we say the electroweak symmetry
$\mathrm{SU(2)}_{\mathrm{L}}\times\mathrm{U(1)}_Y$ is \emph{spontaneously
broken} to $\mathrm{U(1)}_{\mathrm{EM}}$. 

\begin{figure}
\centering
\includegraphics[width=0.7\textwidth]{figs/theory/MexicanHat.pdf}
\caption{\label{fig:mexicanhat} The shape of the ``Mexican hat''
  potential. The minimum of the potential occurs at
a field value that is not $0$.}
\end{figure}

This mechanism is responsible for generating the masses of the gauge
bosons in the standard model, as can be seen by evaluating the 
covariant derivative in Eqn.~\ref{eqn:Lhiggs},
\begin{equation}
D_{\mu}\Phi = (\partial_{\mu} - i g_2 \frac{\sigma_a}{2}W^a_{\mu} -
ig_1\frac{1}{2}B_{\mu})\Phi ~,
\end{equation}
on the vacuum Higgs field. In this case, the kinetic term is:
\begin{align}
|D_{\mu}\Phi|^2 &= \frac{1}{2} \left|\left (\begin{matrix}\partial_{\mu}
    -\frac{i}{2}(g_2W_{\mu}^3 + g_1B_{\mu})&
    -\frac{ig_2}{2}(W_{\mu}^1-iW_{\mu}^2)\\ 
-\frac{ig_2}{2}(W_{\mu}^1+iW^2_{\mu})&\partial_{\mu}
    +\frac{i}{2}(g_2W_{\mu}^3 - g_1B_{\mu})
  \end{matrix}\right)
                                       \left(\begin{matrix}0\\v\end{matrix}\right)\right |^2 \nonumber\\
& = \frac{1}{8} \left ( g_2^2v^2|W_{\mu}^1+iW^2_{\mu}|^2 +
  v^2|g_2W^3_{\mu}-g_1B_{\mu}|^2 \right ) \nonumber\\
& = m_W^2W_{\mu}^+W^{-\mu} +
\frac{1}{2}m_Z^2Z_{\mu}Z^{\mu}~ + \frac{1}{2}m_{\gamma}^2A_{\mu}A^{\mu},
\end{align}
where we can identify three field combinations, $W_{\mu}^{\pm}$ and
$Z_{\mu}$, which have bilinear mass terms, and a fourth $A_{\mu}$,
which does not:
\begin{align}
W_{\mu}^{\pm} &= \frac{1}{\sqrt{2}}(W_{\mu}^1\mp iW^2_{\mu})~, &m_W &= \frac{1}{2}vg_2~,\\
Z_{\mu} &= \frac{g_2W_{\mu}^3 - g_1B_{\mu}}{\sqrt{g_1^2+g_2^2}}~,&m_Z &= \frac{1}{2}v\sqrt{g_1^2+g_1^2}~,\\
A_{\mu} &= \frac{g_2W_{\mu}^3 + g_1B_{\mu}}{\sqrt{g_1^2+g_2^2}}~,&m_{\gamma} &= 0.
\end{align}
The $\PW^{\pm}$ and $\PZ$ bosons have acquired mass, while the photon
$\Pgg$ remains massless.

Another consequence of this symmetry breaking mechanism is the
emergence of a physical spin-$0$ boson. If we expand the field around
its potential minimum
\begin{align}
\Phi(x)&=
\frac{1}{\sqrt{2}}U(x)\left(\begin{matrix} 0\\v+H(x)\end{matrix} \right)
\end{align}
and write out the terms associated to this field, we find
\begin{equation}
\mathcal L_{\mathrm{Higgs}} \supset \frac{1}{2}(\partial^{\mu}H)^2 -
\lambda v^2 H^2 - \lambda v H^3 - \frac{\lambda}{4}H^4~.
\end{equation}
which means this scalar boson, called the \emph{Higgs boson}, is self-interacting and has a mass squared of $m^2_{h^0} =
2\lambda v^2$ at tree-level.

\section{Fermion Masses}
\label{sec:fermionmasses}
As proposed by Weinberg~\cite{PhysRevLett.19.1264}, fermions acquire
mass through interaction with the $\Phi$ field, which has a nonzero
vev. This is accomplished by adding Yukawa terms to the Lagrangian for
each generation,
\begin{equation}
\mathcal L_{\mathrm{Yukawa}} = - y_e \bar\ell_L\Phi e_R -
y_u\bar q_L\Phi u_R  - y_d\bar q_L\tilde\Phi d_R + (\mathrm{h.c.})~,
\end{equation}
where the doublet $\tilde\Phi = i\sigma_2\Phi^{\ast}$ with hypercharge $Y=-1/2$ is
needed to generate masses for the down-type quarks. Then we can
identify the fermion masses as
\begin{align}
m_e &= \frac{y_e v}{2}&m_u &= \frac{y_u
                                          v}{2}& m_d &= \frac{y_dv}{2}~.
\end{align}
%Neutrino masses can also be accommodated in the SM by adding similar
%terms. 
Besides giving masses to the fermions, the Yukawa terms have another important
consequence: they allow the fermions to affect the
observed mass of the Higgs boson through quantum corrections.

\section{Higgs Mass and Naturalness}
\label{sec:higgsnaturalness}
The leading quantum correction to the Higgs mass squared parameter is due to
the large Yukawa coupling to the top quark, which gives the
top quark its large mass. In an effective field theory approach, where
momenta of virtual particles are cut off at the scale
$\Lambda_{\mathrm{UV}}$, we can compute the top quark's contribution in the SM to leading order,
\begin{fmffile}{higgs}
\begin{align}
m^2_{h^0} &= m^2_{h^0\mathrm{(bare)}} +\Delta(m^2_{h^0}) \\
\Delta(m^2_{h^0}) &= \quad\parbox{20mm}{
\begin{fmfgraph*}(20,20)
\fmfkeep{fermion}
\fmfleft{i} 
\fmfright{o} 
\fmf{dashes}{i,v1}
\fmf{dashes}{v2,o}
\fmf{plain,left,tension=.3,label=$t$}{v1,v2}
\fmf{plain,left,tension=.3}{v2,v1}
\fmfv{label=$h^0$,label.angle=90}{i}
\end{fmfgraph*}} \quad + \quad\cdots\\
&= -\frac{3|y_t|^2}{8\pi^2}\Lambda_{\mathrm{UV}}^2 + \cdots~,
\label{eqn:toploop}
\end{align}
where $\Lambda_{\mathrm{UV}}$ is an ultraviolet momentum cutoff,
usually taken to be the grand unification scale $10^{16} \GeV$ or the Planck scale $10^{19} \GeV$. This
quadratic dependence on $\Lambda_{\mathrm{UV}}$ means the Higgs
mass parameter is sensitive to the physics in the ultraviolet. This sensitivity requires a large fine tuning of $m^2_{h^0\mathrm{(bare)}}$ to explain
why $m^2_{h^0}$, measured to be $(125.09\pm0.24 \GeV)^2$, remains small. This undesirable situation is called the
\emph{naturalness problem}. This conundrum of how a light fundamental
scalar particle can exist in the presence of (presumably) new physics
in the ultraviolet is one of the key motivations for new physics at
the \TeV scale, especially \emph{supersymmetry}.


\section{Supersymmetry}
\label{sec:susy}
Supersymmetry (SUSY) is a proposed symmetry of spacetime that 
introduces a bosonic (fermionic) partner for every fermion
(boson)~\cite{Wess,Golfand,Volkov,Chamseddine,Kane,Fayet,Barbieri,Hall,Ramond}. For
many years, such a symmetry was thought to be impossible since in 1967,
Colman and Mandula~\cite{PhysRev.159.1251} published their no-go theorem that says
internal symmetries, those that act on internal degrees of
freedom like spin, cannot be combined with spacetime symmetries in a
nontrivial way. SUSY evades this theorem because it is based
on a \emph{super Lie algebra}, which may include fermionic symmetries
and anticommutation relations as well as the usual bosonic symmetries
and commutation relations.

Supersymmetric extensions of the SM are compelling 
because they 
\begin{enumerate}[(a)]
\item yield a solution to the naturalness problem, alleviating
the fine-tuning of fundamental
parameters~\cite{Witten:1981nf,Dimopoulos:1981zb,Dine:1981za,Dimopoulos:1981au,Sakai:1981gr,Kaul:1981hi},
explained further in Sec.~\ref{sec:susynaturalness}.
\item exhibit gauge coupling
  unification~\cite{Dimopoulos:1981yj,Marciano:1981un,Einhorn:1981sx,Ibanez:1981yh,Amaldi:1991cn,Langacker:1995fk}, and
\item provide a weakly interacting particle candidate for dark matter~\cite{Ellis:1983ew,Jungman:1995df}.
\end{enumerate}

SUSY can be thought of as an extension of the usual group of spacetime
symmetries, known as the Poincar\'{e} group. This group has
generators related to translation symmetry, $P_m$, and Loretnz
symmetry, $M_{mn} = -M_{nm}$, which form a Lie algebra:
\begin{align}
~[P_m,P_n] &= 0 \nonumber\\
~[P_m,M_{np}] &= i(\eta_{mn}P_p-\eta_{mp}P_n) \nonumber\\
~[M_{mm},M_{pq}] &= i(\eta_{mp}M_{np} - \eta_{np}M_{mq} +
                   \eta_{nq}M_{mp} - \eta_{mq}M_{np} )~.
\label{eqn:poincare}
\end{align}

As shown by Coleman and Mandula~\cite{PhysRev.159.1251}, the only way
to extend this symmetry with a new internal symmetry group $G$ with bosonic
generators $B_r$ and Lie algebra,
\begin{equation}
~[B_r,B_s] = f_{rs}{}^tB_t~,
\end{equation}
with structure functions $f_{rs}{}^t$, is if the extended symmetry
group is simply the direct product $($Poincar\'{e}$)\times G$ with a
trivial Lie algerbra,
\begin{equation}
~[B_r,P_m] = [B_r,M_{mn}] = 0~.
\end{equation}

However, SUSY exploits a loophole in the Coleman-Mandula theorem,
which only considers bosonic symmetry generators, by incorpororating
fermionic symmetry generators $Q$ that generate SUSY transformations,
\begin{equation}
Q|\mathrm{boson}\rangle = |\mathrm{fermion}\rangle, ~~~~
Q|\mathrm{fermion}\rangle = |\mathrm{boson}\rangle~.
\end{equation}
Supersymmetries can be combined with the
spacetime symmetries in a way that mixes the two symmetries, as
exemplified by the super Lie algebra for $\mathcal N=1$
SUSY,
\begin{align}
~\{ Q_{\alpha},\bar Q_{\dot{\beta}}\} &= 2\sigma^m_{\alpha\dot\beta} P_m \nonumber\\
~\{ Q_{\alpha},Q_{\beta}\} &= \{ \bar Q_{\dot\alpha},\bar Q_{\dot\beta}\} = 0\nonumber\\
~[ P_m, Q_{\alpha}] &= [P_m,\bar Q_{\dot\alpha}] = 0~.
\label{eqn:n1susy}
\end{align}
In other words, two SUSY generators can combine to generate a spacetime translation.

The most economical supersymmetric extension of the standard model is the
Minimal Supersymmetric Standard Model (MSSM). For supersymmetric
theories, the Lagrangian can be written in terms of vector superfields
$V(x,\theta,\bar\theta)$ and chiral superfields $\Phi_l(x,\theta,\bar\theta)$ that are functions of
\emph{superspace}, an extension of spacetime that includes
anticommuting Grassmanian variables $\theta$ and
$\bar\theta$. Tab~\ref{tab:susyreps} shows the particles of the
MSSM, described by chiral superfields $Q_i, U_i^c, L_i, E_i^c, H_u,
H_d$, and the representations in which they transform under the MSSM gauge group.
\begin{table}
\centering
\begin{tabular}{c|ccc}
&$\mathrm{SU(3)}_{\mathrm{C}}$&$\mathrm{SU(2)}_{\mathrm{L}}$&$\mathrm{U(1)}_Y$ \\\hline
$Q$ & $\mathbf{3}$ & $\mathbf{2}$ & $1/6$\\
$U^c$ & $\mathbf{\bar 3}$ & $\mathbf{1}$ & $-2/3$\\
$D^c$ & $\mathbf{\bar 3}$ & $\mathbf{1}$ & $1/3$\\
$L$ & $\mathbf{1}$ & $\mathbf{2}$ & $-1/2$\\
$E^c$ & $\mathbf{1}$ & $\mathbf{1}$ & $1$\\\hline
$H_u$ & $\mathbf{1}$ & $\mathbf{2}$ & $1/2$\\
$H_d$ & $\mathbf{1}$ & $\mathbf{2}$ & $-1/2$
\end{tabular}
\caption{\label{tab:susyreps} Table summarizing the
    representations in which the chiral superfields transform under
    the MSSM gauge group. See Tab.~\ref{tab:representations} for a
    description of the columns.}
\end{table} 

The MSSM Lagrangian can be split into three terms, the K\"{a}hler potential $K(\Phi_l,\Phi_l^{\dagger},V)$,
which describes the kinetic and gauge-covariant terms, a superpotential $W(\Phi_l)$, which
describes the mass and interaction terms, and a gauge kinetic term $G(V)$,
\begin{align}
\mathcal L_{\mathrm{MSSM}} &= \int \mathrm{d}^4\theta
K(\Phi_l,\Phi_l^{\dagger},V) + \left (\int
 \mathrm{d}^2\theta W(\Phi_l) + \mathrm{h.c.} \right) + \left (\int
\mathrm{d}^2\theta G(V) +
\mathrm{h.c.}\right )\nonumber\\
&= \int \mathrm{d}^4\theta
\sum_{V}
\Phi_{l}^{\dagger}e^{gV}\Phi_{l} + \left (\int
 \mathrm{d}^2\theta W(\Phi_l) + \mathrm{h.c.} \right) + \left (\int
\mathrm{d}^2\theta \frac{1}{4}\mathcal W^{\alpha}\mathcal W_{\alpha} +
\mathrm{h.c.}\right )~,
\label{eqn:mssmlag}
\end{align}
where $\mathcal W_{\alpha}$ is the vector superfield strength\footnote{The vector superfield strength is $\mathcal W_{\alpha} =
  -\frac{1}{4} \bar{\mathcal{D}}^2(e^{-V}\mathcal{D}_{\alpha}
  e^{V})$.} and the index $l$ runs over all the matter superfields
$\Phi_l = Q_i, U_i^c, L_i, E_i^c, H_u, H_d$. The superpotential can be
split into two components based on $R$-parity, a discrete $Z_2$
symmetry often assumed in SUSY model building, defined for each
particle as 
\begin{equation}
P_R = (-1)^{\mathrm{B}-\mathrm{L})+2s}
\end{equation}
where $\mathrm{B}$ is the baryon number, $\mathrm{L}$ is the lepton number, and $s$ is the spin
of the particle. With this assignment, all SM particles have even
$R$-parity ($P_R=+1$), while the superpartners have odd $R$-parity
($P_R=-1$). The $R$-parity conserving terms in the superpotential are
\begin{equation}
W_{\mathrm{RPC}} = - U^c\mathbf{y_u} Q H_u + D^c\mathbf{y_d}QH_d  +
E^c\mathbf{y_e} L H_d +
\mu H_uH_d~,
\label{eqn:Wrpc}
\end{equation}
and the $R$-parity violating terms are
\begin{equation}
W_{\mathrm{RPV}} =\frac{1}{2}\lambda^{ijk}L_iL_jE_k^c +
\lambda^{\prime ijk} L_iQ_jD_k^c + \mu^{\prime i}H_uL_i +
\frac{1}{2}\lambda^{\prime\prime ijk}U_i^cD_j^cD_k^c~.
\label{eqn:Wrpv}
\end{equation}

If $R$-parity is conserved, there are three important phenomenological consequences:
\begin{itemize}
\item The lightest superparter, called the ``lightest
  supersymmetric particle'' or LSP, must be stable. 
\item Each superpartner other than the LSP must eventually decay into a
  state that contains an odd number of LSPs (usually just one).
\item In collider experiments, superpartners can only be produced in even numbers (usually two).
\end{itemize}

\section{Elecroweak SUSY Sector}
\label{sec:ewksusy}
The superpartners of the neutral gauge bosons (neutral gauginos), and
the fermionic partners of the neutral Higgs bosons (neutral higgsinos),
$\psi^0 =(\PSB,\PSWz,\PSHdz,\PSHuz)$, mix to form the
\emph{neutralinos} $(\chiz_1, \chiz_2, \chiz_3, \chiz_4)$. Similarly, the superpartners of the charged gauge bosons
(charged gauginos) and the fermionic partners of the charged Higgs
bosons (charged higgsinos), $\psi^{\pm}=(\PSWp,\PSHup,\PSWm, \PSHdm)$ mix to form the charginos
$(\chip_1, \chip_2, \chim_1, \chim_2)$. In this gauge-eigenstate basis, the neutralino
and chargino mass terms in the Lagrangian are
\begin{equation}
\mathcal L_{\mathrm{MSSM}} \supset -\frac{1}{2}\psi^{0T}\mathbf{M}_N \psi^0 -\frac{1}{2}\psi^{\pm T}\mathbf{M}_C \psi^{\pm} + \mathrm{h.c.}
\end{equation}
with the neutralino mass matrix,
\begin{equation}
\mathbf{M}_N =\left (  \begin{matrix}
M_1 & 0 & -m_Zs_Wc_{\beta} & m_Zs_Ws_{\beta} \\
0& M_2 & -m_Zc_Wc_{\beta} & -m_Zc_Ws_{\beta} \\
-m_Zs_Wc_{\beta}& m_zc_Wc_{\beta} & 0 & -\mu\\
m_Zs_Ws_{\beta}& -m_zc_Ws_{\beta} & \mu & 0
\end{matrix}\right)~,
\end{equation}
and the chargino mass matrix,
\begin{equation}
\mathbf{M}_C =\left (  \begin{matrix}
0 & \mathbf{X}^T \\
 \mathbf{X}& 0
\end{matrix}\right), ~\mathrm{where} ~ \mathbf{X} = \left (  \begin{matrix}
M_2 & \sqrt{2}m_Ws_{\beta}\\
 \sqrt{2}m_Wc_{\beta}& \mu
\end{matrix}\right)~,
\end{equation}
where $M_2$ and $M_1$ are the wino and bino mass terms, respectively.
The eigenvalues of these matrices are the masses of the
neutralinos and charginos.

\section{SUSY and Naturalness}
\label{sec:susynaturalness}

For SUSY to provide a ``natural'' solution to the gauge hierarchy problem,
the top squark, bottom squark, and gluino must have masses below a few
TeV, making them accessible at the CERN LHC. 

To make this quantitative, we first need to define a measure of
fine-tuning, such as 
\begin{equation}
\Delta = \frac{\Delta(m^2_{h^0})}{m^2_{h^0}}
\end{equation}
Other measures of fine-tuning are possible, such as those in...

In the decoupling limit of the MSSM at tree-level, we have
\begin{equation}
m^2_{h^0} = 2(m_{H_u}^2 + |\mu|^2) =  m_Z^2\cos^2(2\beta)  ~.
\end{equation}
Due to this tree-level relation between $\mu$, which controls the
masses of the higgsinos, and $m^2_{h^0}$, we expect the higgsinos to
be light in order to keep $m^2_{h^0}$ small.

When $\mu$ is small ($\mu \ll M_1, M_2$), the two lightest
neutralinos and the lightest chargino are higgsino-like and their
masses are close to each other,
\begin{align}
m_{\chiz_1} &= |\mu| + \frac{m_Z^2(1+s_{\beta})(\mu - M_1c_W^2-M_2s_W^2)}{2(\mu-M_1)(\mu-M_2)} + \cdots\\
m_{\chiz_2} &= |\mu| + \frac{m_Z^2(1-s_{\beta})(\mu + M_1c_W^2+M_2s_W^2)}{2(\mu+M_1)(\mu+M_2)} + \cdots\\
m_{\chipm_1} &= |\mu| - \frac{m_W^2(\mu + M_2\sin 2\beta)}{M_2^2-\mu^2} + \cdots~.
\end{align}
Fig.~\ref{fig:neutralinos} shows the mass differences
$m_{\chipm_1}-m_{\chiz_1}$ and $m_{\chiz_2}-m_{\chiz_1}$ as a function of the wino mass $M_2$.

\begin{figure}[tb!]
\centering
\includegraphics[width=0.8\textwidth,clip=true,viewport= 0 30 610 450]{figs/theory/neutralinos.pdf}
\caption{The mass difference between the lightest chargino and the
  lightest neutralino as a function of the wino mass $M_2$
  assuming $\tan\beta=10$, $\mu=200 \GeV$ and $M_1 = 3 \TeV$.\label{fig:neutralinos}}
\end{figure}

There are additional considerations at loop level. In particular the
top squark has a large radiative correction which partially cancels
with the contribution from the top quark in Eqn.~\ref{eqn:toploop}:
\begin{align}
\Delta(m^2_{h^0}) &= \quad\parbox{20mm}{\fmfreuse{fermion}} \quad + \quad
\parbox{20mm}{
\begin{fmfgraph*}(20,20)
\fmfkeep{boson}
\fmfleft{i} 
\fmfright{o} 
\fmf{dashes}{i,v}
\fmf{dashes,right,tension=0.7,label=$\tilde t_{1,,2}$}{v,v}
\fmf{dashes}{v,o}
\fmfv{label=$h^0$,label.angle=90}{i}
\end{fmfgraph*}}
 \quad + \quad
\parbox{20mm}{
\begin{fmfgraph*}(20,20)
\fmfkeep{sunset}
\fmfleft{i}
\fmfright{o}
\fmf{dashes}{i,v1}
\fmf{dashes}{v2,o}
\fmf{dashes,left,tension=.3,label=$\tilde t_{1,,2}$}{v1,v2}
\fmf{dashes,left,tension=.3}{v2,v1}
\fmfv{label=$h^0$,label.angle=90}{i}
\end{fmfgraph*}} \quad + \quad\cdots\\
 &= -\frac{3|y_t|^2}{8\pi^2}\Lambda_{\mathrm{UV}}^2 +
  \sum_{i=1}^{2} \left ( \frac{3|y_t|^2}{16\pi^2}\Lambda_{\mathrm{UV}}^2 - \frac{3|y_t|^2m_{\tilde
  t_i}^2}{8\pi^2}\log\left (\frac{\Lambda_{\mathrm{UV}}}{m_{\tilde
   t_i}}\right ) \right )+ \cdots \\
 &= -\sum_{i=1}^{2} \left ( \frac{3|y_t|^2m_{\tilde
  t_i}^2}{8\pi^2}\log\left (\frac{\Lambda_{\mathrm{UV}}}{m_{\tilde
   t_i}}\right ) \right )+ \cdots~,
\end{align}
Naturalness therefore requires, very roughly, $m_{\tilde t_{i}}\lesssim
400 \GeV$ if $\Lambda_{\mathrm{UV}}\approx
10\TeV$~\cite{Brust:2011tb}. A more complete treatment takes into
account top squark mixing, 
\begin{equation}
\Delta(m^2_{h^0}) = ...
\end{equation}

Two-loop contribution from gluinos,
\begin{align}
\Delta(m^2_{\tilde t_i}) &= \quad 
\parbox{20mm}{
\begin{fmfgraph*}(20,20)
\fmfkeep{gluinosunset}
\fmfleft{i}
\fmfright{o}
\fmf{dashes}{i,v1}
\fmf{dashes}{v2,o}
\fmf{gluon,left,tension=.3,label=$\tilde g$}{v1,v2}
\fmf{plain,left,tension=.3,label=$t$}{v2,v1}
\fmfv{label=$\tilde t_{i}$,label.angle=90}{i}
\fmffreeze
\fmf{plain,left,tension=.3}{v1,v2}
\end{fmfgraph*}} + \quad \cdots\\
&= \frac{2g_s^2m_{\tilde g}^2}{3\pi^2}\log \left
  (\frac{\Lambda_{\mathrm{UV}}}{m_{\tilde g}} \right) + \cdots~,
\end{align}
\end{fmffile}


Taking these factors into consideration, we have three conditions for natural
SUSY~\cite{naturalSUSY}:
\begin{enumerate}[(a)]
\item quasi-degenerate higgsino-like chargino $\chipm_1$ and two
  neutralinos $\chiz_1, \chiz_2$, below
  $200 - 350 \GeV$, 
\item two stops $\sTop_{1,2}$ and one (left-handed) $\sBot_1$ below
  $500-700 \GeV$, and
\item a gluino below $900 - 1500 \GeV$.
\end{enumerate}

\section{Simplified Natural Supersymmetry Models}
\label{sec:sms}

\begin{figure}[htb!]
\centering
\includegraphics[width=0.85\textwidth]{figs/analysis8TeV/naturalSpectrum.pdf}
\caption{\label{fig:spectrum} The simplified natural SUSY spectrum
  considered in this paper, along with the assumed decay modes.}
\end{figure}

In Ch.~\ref{ch:analysis8TeV} and Ch.~\ref{ch:analysis13TeV}, natural simplified SUSY scenarios are used to interpret
results. The LSP is the lightest neutralino $\chiz_1$ while the NLSP
is the lightest chargino $\chipm_1$.  They are both higgsinos and
their mass splitting is taken to be 5\GeV. The NLSP decays to the LSP
and a virtual $\PW$ boson ($\chipm_1 \to \PW^{\ast} \chiz_1$). The
other SUSY particles accessible at the LHC are the gluino and the
lightest top and bottom squarks. All other SUSY particles are
assumed to be too heavy to participate in the interactions. The SUSY
particles and their possible decay modes within this natural SUSY
spectrum are summarized in Fig.~\ref{fig:spectrum}.

In the context of this natural spectrum, several simplified
models~\cite{ArkaniHamed:2007fw,Alwall:2008ag,Alwall:2008va,Alves:2011sq,Alves:2011wf,Graesser:2012qy}
are considered for gluino pair production, based on three-body gluino
decays, in which each gluino decays to one of the following final states~\cite{SUS-11-016}:
\begin{itemize}
\item a top quark (antiquark) and a bottom antiquark (quark),
  and the NLSP; 
\item a top quark-antiquark ($\ttbar$) pair and the LSP;
\item a bottom quark-antiquark ($\bbbar$) pair and the LSP.
\end{itemize}
Furthermore, we separately consider the case in which each gluino
decays to
\begin{itemize}
\item a first or second generation quark-antiquark ($\qqbar$) pair and the LSP.
\end{itemize}

\begin{figure*}[thb!]
\centering
\includegraphics[width=0.32\textwidth]{figs/theory/T1bbbb.pdf}
\includegraphics[width=0.32\textwidth]{figs/theory/T1tbbb.pdf}
\includegraphics[width=0.32\textwidth]{figs/theory/T1ttbb.pdf} \\
\includegraphics[width=0.32\textwidth]{figs/theory/T1tbtb.pdf} 
\includegraphics[width=0.32\textwidth]{figs/theory/T1tttb.pdf}
\includegraphics[width=0.32\textwidth]{figs/theory/T1tttt.pdf} \\
\includegraphics[width=0.32\textwidth]{figs/theory/T1qqqq.pdf} \\
\includegraphics[width=0.32\textwidth]{figs/theory/T2bw.pdf}
\includegraphics[width=0.32\textwidth]{figs/theory/T2tb.pdf}
\includegraphics[width=0.32\textwidth]{figs/theory/T2tt.pdf}
\caption{Diagrams displaying the event topologies of gluino (upper 7
  diagrams) and top-squark (lower 3 diagrams) pair production
  considered in this thesis.\label{fig:SMSDiagrams}}
\end{figure*}

In addition, several simplified models are considered for
the production of top-squark pairs, in which each top squark decays to
one of the following final state:
 \begin{itemize}
\item a bottom quark and the NLSP;
\item a top quark and the LSP.
\end{itemize}

The corresponding Feynman diagrams are shown in
Fig.~\ref{fig:SMSDiagrams}.

\subsection{Technical Implentation in \PYTHIA}
To simplify the treatment of the sparticle decays in \PYTHIA v6.4.26, we directly implement three-body decays of
the form $\chipm_1 \to \chiz_1 f f^{\prime}$, with branching ratios
as shown in Table~\ref{tab:nlspbr}.
\begin{table}
\centering
\begin{tabular}{l|r}
decay mode & branching ratio \\\hline
$\chip_1 \to \chiz_1 u \bar d$ &  35.1\%\\
$\chip_1 \to \chiz_1 c \bar s$ &  35.1\%\\
$\chip_1 \to \chiz_1 e^+ \nu_e$ &  11.7\%\\
$\chip_1 \to \chiz_1 \mu^+ \nu_{\mu}$ &  11.7\%\\
$\chip_1 \to \chiz_1 \tau^+ \nu_{\tau}$ &  6.4\%
\end{tabular}
\caption{\label{tab:nlspbr}Table of branching ratios implemented in
  \PYTHIA v6.4.26 for the NLSP
  $\chipm_1$ in the simplified natural SUSY model considered in this chapter.}
\end{table}