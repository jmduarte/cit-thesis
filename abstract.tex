\begin{abstract}
In this thesis, we present two inclusive searches for supersymmetric
particles at $\sqrt{s}=8$ and $13\TeV$ using the razor variables and guided
by the principle of naturalness. We build a framework to explore the
natural supersymmetry parameter space of gluino and top squark masses
and branching ratios, which is a unique attempt to cover this
parameter space in a more complete way than ever before using LHC data. With this approach, the production of top squarks and
gluinos are excluded below $\sim700\GeV$ and $\sim1.6\TeV$, respectively,
independent of the branching ratios, constituting one of the
tightest constraints on natural supersymmetry from the LHC. Motivated by the
need to mitigate the effects of multiple interactions per bunch
crossing (pileup), an essential feature of present and future hadron colliders, in this thesis we also study the precision timing capabilities of a LYSO-based sampling calorimeter, and achieve a time
resolution of $\sim30\unit{ps}$ in electron test beam measurements. The
achieved resolution corresponds to the precision needed to significantly reduce the inclusion of pileup particles in the
reconstruction of the event of interest. This study is foundational in
building an R\&D program on precision timing for the high-luminosity LHC and other
future hadron colliders. We also propose alternative simplified models to study
Higgs-plus-jets events at the LHC, and reinterpret an excess
observed at $8\TeV$ in the context of these models. Finally, we discuss a search for narrow resonances
in the dijet mass spectrum at $13\TeV$ using the data-scouting
technique at CMS, which records a smaller event format to increase the
maximum recordable rate. For the benchmark models with a vector or axial-vector
mediator that couples to quarks and dark matter particles, the dijet
search excludes mediator masses from $0.5\TeV$
up to $\sim2.7\TeV$ largely independent of the dark matter particle
mass, which constitutes a larger exclusion than traditional
mono-X searches at the LHC. In the plane of the
dark matter-nucleon interaction cross section versus dark matter mass,
the dijet search is also more sensitive than direct
detection experiments for spin-dependent cross sections. 
\end{abstract}
