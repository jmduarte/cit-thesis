\begin{abstract}
In this thesis, we present two inclusive searches for supersymmetric
particles at $8\TeV$ and $13\TeV$ guided by the principle of naturalness. We build a framework to explore the
natural supersymmetric parameter space of gluino and top squark branching
ratios, which represents a unique attempt to cover this parameter
space in a more complete way than ever before. With this approach, the production of top squarks and
gluinos are excluded below $\sim700\TeV$ and $\sim1.6\TeV$, respectively,
independent of the branching ratios. We also propose alternative simplified models to study
$\PH+\mathrm{jet}$ events at the LHC, and reinterpret an excess
observed at $8\TeV$ in the context of these models. Motivated by the
need to mitigate effects of multiple interactions per bunch
crossing (pileup), an essential feature of present and future
colliders, we study the precision timing capabilities of a LYSO-based sampling calorimeter, and achieve a time
resolution of $\sim30\unit{ps}$ in electron test beam measurements. The
achieved resolution corresponds to the precision needed to significantly reduce the inclusion of pileup particles in the
reconstruction of the event of interest. We discuss a search for narrow resonances
in the dijet mass spectrum at $13\TeV$ using the data-scouting
technique at CMS, which records a smaller event format to increase the
maximum recordable rate. For models with a vector or axial-vector
mediator that couples to quarks and dark matter particles, the dijet
search is excludes a larger range of ($m_{mathrm{med}}$,$m_{\mathrm{DM}}$) parameter space 
traditional $\MET+\mathrm{X}$ searches at the LHC.
\end{abstract}
