\chapter{Attribution}
\label{ch:attribution}
Ch.~\ref{ch:fundamentals}~and~Ch.~\ref{ch:naturalness} lays out the theoretical foundations of particle physics, supersymmetry, and naturalness, and represents the work of theoretical physicists cited therein. Sec.~\ref{sec:sms} presents a framework for interpretation proposed and established by myself in collaboration the CMS SUSY group, especially Maurizio Pierini and Frank Wuerthwein. Ch.~\ref{ch:lhc} gives an overview of the LHC and represents the work of many engineers and physicists cited therein.

Ch.~\ref{ch:cms} recounts work done by my CMS collaborators on the CMS detector. An exception is the discussion of the alignment and calibration triggers and the HLT conditions validation in Sec.~\ref{sec:alca}, which I coordinated for the start of Run 2 in 2015.  In collaboration with Alex Mott, Si Xie, Dustin Anderson, and Maurizio Pierini, Ch.~\ref{ch:hlt13TeV} presents my work on the development of the 13 \TeV razor triggers. Ch.~\ref{ch:analysis8TeV} presents the 8\TeV inclusive razor search for SUSY, in which I was the lead analyst in collaboration with Maurizio Pierini and Maria Spiropulu. The combination of the hadronic razor search with the exclusive single-lepton top squark search in Sec.~\ref{sec:interp} was performed in collaboration with Frank Golf and Ryan Ward Kelley. Ch.~\ref{ch:analysis13TeV} describes the 13\TeV inclusive razor search for SUSY and represents my work for the fit-based background estimation method B, the work of Dustin Anderson for the simulation-based background estimation method A, as well as the work of Jay Lawhorn, Cristi\'{a}n Pe\~{n}a, Si Xie, Artur Apresyan, Maurizio Pierini, and Maria Spiropulu. Ch.~\ref{ch:timing} discusses test beam studies of the timing performance of LYSO-based calorimeters, done by myself in collaboration with Adi Bornheim, Cristi\'{a}n Pe\~{n}a, Dustin Anderson, Si Xie, Artur Apresyan, Anatoly Ronzhin, Jason Trevor, and Maria Spiropulu.

App.~\ref{ch:pheno} represents the work done by myself, Cristi\'{a}n Pe\~{n}a, Maurizio Pierini, and Maria Spiropulu. Finally, I was the lead analyst in the 13\TeV search for narrow resonances in the dijet mass spectrum presented in App.~\ref{ch:dijet}, although I collaborated closely with colleagues in the CMS Exotica group, including Dustin Anderson, Si Xie, Artur Apresyan, Maurizio Pierini, David Sheffield, Juska Pekkanen, Robert Harris, and Francesco Santanastasio. In preparing the $\PZpr_\mathrm{B}$ and dark matter interpretation of the dijet searches, I also collaborated with Bora Isildak, Zhixing ``Tyler'' Wang, Phil Harris, Nhan Tran, Tristan Du Pree, and others.