\chapter{Searches for Supersymmetry at $\sqrt{s}=13\TeV$}

The use of razor variables to search for
supersymmetry (SUSY) was established during the LHC Run I by several
studies, both by the CMS~\cite{razor2010,razorPRL,razorPRD,razor8TeV} and
ATLAS~\cite{Aad:2012naa,ATLAS-dilepton} collaborations. 

In this chapter, we describe an inclusive search using the LHC 2015 dataset
accumulated by the CMS experiment at a center-of-mass energy
$\sqrt{s}=13\TeV$ with an integrated luminosity of 2.1 \fbinv. Given
the avaialable integrated luminosity and 13 \TeV production cross
sections expected, the targeted model is the pair production of heavy gluinos. To probe
a broad range of signal models in a variety of final states, we perform
the search in multiple event categories, based on the presence of
specific reconstructed physics objects, such as electrons, muons, and
b-jets.  With this strategy the purity of specific classes of signals is enhanced in
specific event categories while the SM backgrounds are generally suppressed. 

In section~\ref{sec:razor} we describe the razor variables. In section~\ref{sec:evtsel}, the details of the
object identification and event selection requirements are presented, as well as the
SUSY signal Monte Carlo samples used in the interpretation of the results. The background, estimated 
in each event category using a two-dimensional maximum likelihood fit to the distribution of the data in
the razor variables is discussed in section~\ref{sec:bgr}. In section~\ref{sec:sysuncertainties} 
the systematic uncertainties on the signal
 are discused, and in section~\ref{sec:results} we present the search
 results. We find that the observed data is
 compatible with the background-only hypothesis, and set exclusion limits on specific signal models described in section~\ref{sec:evtsel}.


\section{Binned Likelihood}

\begin{equation}
\hat\nu_i = \int \int dM_R dR^2 f(M_R,R^2)
\end{equation}
