\chapter{The Compact Muon Solenoid Detector}

The central feature of the CMS detector is a
superconducting solenoid of 6\unit{m} internal diameter, providing a
magnetic field of 3.8\unit{T}. Within the superconducting solenoid
volume are a silicon pixel and a silicon strip tracker, a
lead-tungstate crystal electromagnetic calorimeter, and a
brass/scintillator hadron calorimeter, each composed of a barrel and
two endcap sections. Muons are measured in gas-ionization detectors
embedded in the magnet steel flux-return yoke outside the
solenoid. Extensive forward calorimetry complements the coverage
provided by the barrel and endcap detectors. Jets and leptons are
reconstructed within the pseudorapidity region $\abs{\eta}<3$, covered by the
electromagnetic and hadron calorimeters. Muons are reconstructed with
$\abs{\eta}<2.4$. Events are selected by a
two-level trigger system. The first level (L1) is based on a hardware
filter, followed by a software-based high level trigger (HLT). A more
detailed description of the CMS detector, together with a definition
of the coordinate system used and the relevant kinematic variables,
can be found in Ref.~\cite{Adolphi:2008zzk}.

\section{Alignment and Calibration}
\section {High Level Trigger}
