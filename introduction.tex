\chapter{Introduction}
\label{ch:intro}

The standard model (SM) of particle physics is a description of nature
composed of two main ingredients:
\begin{enumerate}[(a)]
\item a list of known particles and their quantum properties, and
\item an inventory of the interactions of these particles through the fundamental forces
(except for gravity): the strong nuclear force, the weak nuclear force, and the electromagnetic
force. 
\end{enumerate}
The discovery of the Higgs boson, announced by the CMS and ATLAS
collaborations on July 4, 2012~\cite{CMShiggs,ATLAShiggs}, was
thought to be the final missing piece that completes and confirms the
SM as a description of nature. However, several theoretically- and
experimentally-motivated questions remain unanswered. Three
of the most important questions are:
\begin{enumerate}[(a)]
\item What is the origin of \emph{dark matter} whose existence is
  inferred from measurements of galactic rotation curves~\cite{1980ApJrotationcurves,1989HIrotationcurves} and weak gravitational lensing
  observations of the ``Bullet cluster''~\cite{Clowe:2006eq}?
\item Do the strong nuclear, weak nuclear, and electromagnetic
  forces unify at a high energy scale?
\item How does a fundamental scalar like the Higgs boson remain
  ``light'' when its mass is sensitive to new particles at high energy
  scales through radiative corrections?
\end{enumerate}

In the 1970s, \emph{supersymmetry} was proposed as a
possible extension of spacetime symmetry that relates fermions and
bosons~\cite{Ramond,Golfand,Volkov,Wess,Fayet}. Little
more than a curiosity at first, supersymmetry
emerged as the leading candidate for a theory of physics beyond the
standard model in the 1980s and 1990s, as it was understood that this
one principle may provide an economical and beautiful solution to
three pressing issues in the SM by (a) providing a weakly interacting particle candidate for dark
matter~\cite{Ellis:1983ew,Jungman:1995df}, (b) exhibitting gauge coupling
unification~\cite{Dimopoulos:1981yj,Marciano:1981un,Einhorn:1981sx,Ibanez:1981yh,Amaldi:1991cn,Langacker:1995fk},
and (c) alleviating the fine-tuning of the Higgs mass~\cite{Witten:1981nf,Dimopoulos:1981zb,Dine:1981za,Dimopoulos:1981au,Sakai:1981gr,Kaul:1981hi}.


This thesis focuses on efforts to discover supersymmetric
particles in collider searches and is divided into five parts. In
Part~\ref{part:intro}, the SM, supersymmetry, and naturalness are
introduced, motivating the search for natural SUSY. 

Part~\ref{part:searches} begins with a description of the CMS
experiment and the
LHC. Chapters~\ref{ch:analysis8TeV}~and~\ref{ch:analysis13TeV} present
two searches for natural SUSY performed at $\sqrt{s}=8
\TeV$ and $13 \TeV$, which together represent a unique attempt to cover the phase space of
natural SUSY in a more complete way than ever before. Finally,
Chapter~\ref{ch:pheno} studies an excess observed in data at $\sqrt{s}=8
\TeV$ and proposes alternative simplified natural SUSY models to study
at $\sqrt{s}=13 \TeV$. 

Part~\ref{part:timing} motivates probing the \TeV and multi-\TeV scale
with high-luminosity colliders as ``the final hiding place'' of natural SUSY. Chapter~\ref{ch:timing} discusses the
requirements for detectors at future high-luminosity colliders,
specifically the need for calorimeters with precision
timing capabilities due to the high rate of simultaneous interactions
per bunch crossing (pileup) anticipated. Finally, we present a
proof-of-concept LYSO-based sampling calorimeter, and estimate its
ultimate timing performance capability based on test-beam
measurements.

Although not the main focus of this thesis, Part~\ref{part:app}
describes an important search for exotic new physics in the dijet mass
spectrum at $\sqrt{s}=13 \TeV$. Partially motivated by
the excess at $750 \GeV$ in both the CMS and ATLAS diphoton mass
spectra, the search extends the previous dijet searches by searching
in the low-mass end of the spectrum ($>450 \GeV$) using the
CMS data scouting technique.