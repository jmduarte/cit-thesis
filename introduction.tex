\chapter{Introduction}
\label{ch:intro}

The standard model (SM) of particle physics is a description of nature
composed of two main ingredients:
\begin{enumerate}[(a)]
\item a list of known particles and their quantum properties, and
\item an inventory of the interactions of these particles through the fundamental forces
(except for gravity): the strong nuclear force, the weak nuclear force, and the electromagnetic
force. 
\end{enumerate}
The discovery of the Higgs boson, announced by the CMS and ATLAS
collaborations on July 4, 2012~\cite{CMShiggs,ATLAShiggs}, was
thought to be one of the final missing pieces\footnote{Two other missing
  pieces are due to the strong CP problem, which demands the existence of
  something like an axion in Peccei-Quinn theory, and nonzero neutrino masses, which require the addition of a right-handed neutrinos.} that completes and confirms the
SM as a description of nature. However, several theoretically- and
experimentally-motivated questions remain unanswered. Three
of the most important questions are:
\begin{enumerate}[(a)]
\item What is the origin of \emph{dark matter} whose existence is
  inferred from measurements of galactic rotation curves~\cite{1980ApJrotationcurves,1989HIrotationcurves} and weak gravitational lensing
  observations of the ``Bullet cluster''~\cite{Clowe:2006eq}?
\item Do the strong nuclear, weak nuclear, and electromagnetic
  forces unify at a high energy scale?
\item How does a fundamental scalar like the Higgs boson remain
  ``light'' when its mass is sensitive to new particles at high energy
  scales through radiative corrections?
\end{enumerate}

In the 1970s, \emph{supersymmetry} was proposed as a
possible extension of spacetime symmetry that relates fermions and
bosons~\cite{Ramond,Golfand,Volkov,Wess,Fayet}. Little
more than a mathematical curiosity at first, supersymmetry
emerged as the leading candidate for a theory of physics beyond the
standard model in the 1980s and 1990s, as it was understood that this 
principle may provide an economical and beautiful solution to
three pressing issues in the SM by (a) providing a weakly interacting particle candidate for dark
matter~\cite{Ellis:1983ew,Jungman:1995df}, (b) exhibiting gauge coupling
unification~\cite{Dimopoulos:1981yj,Marciano:1981un,Einhorn:1981sx,Ibanez:1981yh,Amaldi:1991cn,Langacker:1995fk},
and (c) alleviating the fine-tuning of the Higgs mass~\cite{Witten:1981nf,Dimopoulos:1981zb,Dine:1981za,Dimopoulos:1981au,Sakai:1981gr,Kaul:1981hi}.

This thesis focuses on efforts to search for supersymmetric
particles in collider searches and is divided into five parts. It is
the story of how we examined the principal motivation for supersymmetry,
identified the most relevant event topologies based on this guiding
principle, and exhaustively searched the supersymmetric phase space
using an sensitive basis of observables, known as the \emph{razor variables}.

In Part~\ref{part:intro}, the SM, supersymmetry, and naturalness are
introduced, motivating the search for natural SUSY. The razor approach
to supersymmetric kinematics is detailed in
Sec.~\ref{sec:kinematic}. Part~\ref{part:lhccms} describes the LHC and
the CMS experiment, especially those aspects critical to the searches,
such as the high-level trigger. In Part~\ref{part:searches}, chapters~\ref{ch:analysis8TeV}~and~\ref{ch:analysis13TeV} present
two searches for natural SUSY performed at $\sqrt{s}=8
\TeV$ and $13 \TeV$, which together represent a unique attempt to cover the phase space of
natural SUSY in a more complete way than ever before. 

Part~\ref{part:timing} motivates probing the \TeV and multi-\TeV scale
with high-luminosity colliders as ``the final hiding place'' of
natural SUSY. Chapter~\ref{ch:timing} discusses the
requirements for detectors at future high-luminosity colliders,
specifically the need for calorimeters with precision
timing capabilities due to the high rate of simultaneous interactions
per bunch crossing (pileup) anticipated. Finally, we present a
proof-of-concept LYSO-based sampling calorimeter, and estimate its
ultimate timing performance capability based on test-beam
measurements.

Part~\ref{part:app} is composed of two appendices related to other
searches for new physics at the LHC. In
Chapter~\ref{ch:pheno}, we reinterpret an excess observed in
$\PH+\mathrm{jet}$ events at $\sqrt{s}=8\TeV$ and propose alternative
simplified SUSY models to study this final state at $\sqrt{s}=13 \TeV$. Chapter~\ref{ch:dijet} describes an important search for exotic new physics in the dijet mass
spectrum at $\sqrt{s}=13 \TeV$ with far-reaching consequences for many
different models. Partially motivated by the excess at $750 \GeV$ in both the CMS~\cite{Khachatryan:2016hje} and ATLAS~\cite{Aaboud:2016tru} diphoton mass
spectra\footnote{More recent results using additional data collected
  at in 2016 from CMS~\cite{CMS-PAS-EXO-16-027} and
  ATLAS~\cite{ATLAS-CONF-2016-059} suggest the diphoton excess at $750 \GeV$ may have been a statistical fluctuation.}, the search extends the previous dijet searches by searching
in the low-mass end of the spectrum ($>450 \GeV$) using the
CMS data scouting technique. The dijet resonance search is also interpreted in the context of dark matter (DM) production at the LHC. In
this scenario, the search is sensitive to the production of a vector
or axial-vector mediator that couples to quarks and DM
particles.