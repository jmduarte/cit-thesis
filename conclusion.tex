\chapter{Conclusion and Outlook}
\label{ch:conclusion}
In this thesis, we first presented the theoretical 
%and experimental: maybe add in more experimental motivations: proton
%decay, dm ?
motivations for a specific supersymmetric extension of
the standard model in which the quadratic divergence of the Higgs
mass is tamed to merely a logarithmic one. One interpretation of this
resulting sensitivity is that the fundamental Higgs mass parameter no longer
needs to be finely-tuned for the physical Higgs mass to match the
observed value of $125\GeV$, yielding a more natural theory. The particle spectrum of this
scenario includes light higgsino-like neutralinos and charginos, a light
bottom and top squark, and a light gluino, all accessible at the
current LHC energy of $13 \TeV$. 

We reviewed the LHC and the CMS detector. In particular, we focused on
the aspects of CMS important for the natural SUSY searches, including the
silicon tracker, the electromagnetic and hardon calorimeters, the
high-level trigger, the alignment and calibration, the particle-flow
reconstruction of jets and missing transverse momentum, and the
identification of jets originating from \Pqb-quarks. We also discussed
the topological HLT paths based on razor variables developed for $13\TeV$ searches in hadronic final states.

We then described two inclusive searches for natural SUSY using the razor variables
performed at $\sqrt{s}=8\TeV$ and $13\TeV$, respectively. These
searches go beyond the ``simplified model'' paradigm by interpreting results in a broader natural SUSY
context, with multiple gluino or top squark decay modes considered simultaneously
%corresponding to different branching ratios for the gluino or top squark. 
The analyses exclude a top squark below $\sim700 \TeV$ and a
gluino below $\sim1.6\TeV$ for a low-mass neutralino LSP and
independent of the branching ratios. This sensitivity to different
gluino and top squark decay modes is a consequence of the inclusive approach of the
searches, which consider events with $0$, $1$, and $\geq2$ leptons, and $0$, $1$, $2$,
and $\geq3$ \cPqb-tagged jets.

As continuing the search for natural SUSY requires colliders going to
higher luminosities and therefore pileup, we discussed the possibility of using calorimeters with precision timing
capabilities to mitigate to the problems associated with the higher
pileup anticipated after the HL-LHC upgrade. In dedicated experiments
at the FNAL test beam facility, we obtained a time resolution of $\sim30\unit{ps}$ using a
LYSO-based calorimeter and different light propagation experimental
setups in an electron test beam. The achieved time resolution corresponds to the precision
needed to significantly reduce (by about a factor of ten) the inclusion of pileup particles in the
reconstruction of the event of interest at the HL-LHC.

In an appendix, we proposed two simplified model topologies for searches for SUSY in
$\PH(\Pgg\Pgg)+$jet events. We reinterpreted the results of an
$8\TeV$ CMS search for SUSY using razor variables, and found both models to be
consistent with the excess observed in data at a
$\sim2\sigma$-level. An updated $13\TeV$ search from CMS used one of
the models proposed and also found an excess in data consistent with one of
the model topologies at the $\sim2\sigma$-level at the same mass values.

Finally, in another appendix, we discussed a search for narrow resonances
in the dijet mass spectrum at $13\TeV$ using the data-scouting
technique at CMS, which records a smaller event format to increase the
maximum recordable rate. This search has far-reaching implications for many
models of new physics, including models with a vector or axial-vector
mediator that couples to quarks and dark matter particles. 
For the benchmark choice of mediator couplings, the dijet
search excludes mediator masses from $0.5\TeV$
up to $\sim2.5\TeV$ largely independent of the dark matter particle
mass, which constitutes a larger exclusion than traditional
$\MET+\mathrm{X}$ (mono-X) searches at the LHC. 

Despite the null results in the search for physics beyond the
standard model described in this thesis, the LHC continues to be the most exciting discovery
machine in the world. It remains the best place to seek answers the most pressing
theoretical questions of the day: How is the Higgs mass $125 \GeV$, a value too small for the
standard model without SUSY but too large for many SUSY scenarios? Are
the couplings of the Higgs boson exactly as predicted by the standard model?
Are there weak-scale SUSY particles? Does dark matter couple
(indirectly) to quarks? Because of these unanswered questions, I
believe we are on the precipice of a paradigm-shifting discovery,
which will usher in an era of characterization and measurement of new
particles and couplings, rather than exclusions.

